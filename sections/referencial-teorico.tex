% Destacar o sumário da presente seção antes de iniciá-la
\AtBeginSection[]{
	\begin{frame}
		\frametitle{}
		\tableofcontents[currentsection]
	\end{frame}
}

\section{Referencial Teórico}

\subsection{Usabilidade}
\begin{frame}{Referencial Teórico}
	{Usabilidade}
	\begin{figure}
		\includegraphics[width=9cm]{images/if-salas-pagina.png}
		\caption{Página inicial logada - visão do aluno}
	\end{figure}
\end{frame}

\begin{frame}{Referencial Teórico}
	{Usabilidade}
    %Modelo de lista de itens
	\begin{itemize}[<+->]
		\item Facilidade de Aprendizado: navegação fluida com acesso fácil às turmas, atividades atribuídas com suas respectivas informações e menu lateral com informações úteis.
		\item Eficiência: acesso fácil às ultimas atividades atribuídas, informações e calendário acadêmico.
		\item Satisfação: design limpo com utilização de ícones, cores, emojis e um bom constraste entre os elementos.
		\item Facilidade de Memorização: fluxos simples e bem conhecidos de navegação, como menu lateral e cards informativos.
		\item Erros: outros fluxos de erro são mais vísiveis na aplicação.
	\end{itemize}
\end{frame}

\subsection{Microinterações}
\begin{frame}{Referencial Teórico}
	{Microinterações}
	\begin{figure}
		\includegraphics[width=6cm]{images/senha.png}
		\caption{Preenchimento de senha no cadastro do IF Salas}
	\end{figure}
\end{frame}

\begin{frame}{Referencial Teórico}
	{Microinterações}
    %Modelo de lista de itens
	\begin{itemize}[<+->]
		\item Gatilho: o usuário insere um caractere no campo de senha
		\item Regras: limite de caracteres, letra maiúscula, letra minúscula, caractere especial e número
		\item Feedback: fornecido por meio de dicas visuais como mudança de cores das letras e barra de progresso
		\item Loop: o usuário pode recomeçar a criação da senha a qualquer momento
	\end{itemize}
\end{frame}