% Destacar o sumário da presente seção antes de iniciá-la
\AtBeginSection[]{
	\begin{frame}
		\frametitle{}
		\tableofcontents[currentsection]
	\end{frame}
}

\section{Metodologia}

\subsection{Prototipação}
\begin{frame}{Metodologia}
	{Prototipação}
	\begin{figure}
		\includegraphics[width=6cm]{images/figma-proto.png}
		\caption{Captura de tela do protótipo do IF Salas feito no FIGMA}
	\end{figure}
\end{frame}

\subsection{Codificação e versionamento}
\begin{frame}{Metodologia}
	{Codificação e versionamento}
	\begin{itemize}[<+->]
		\item Desenvolvimento guiado por \href{https://www.artima.com/articles/tracer-bullets-and-prototypes}{Projéteis Luminosos}
		\item Arquitetura e organização dos arquivos do código do projeto foi estruturada seguindo as práticas mais difundidas no mercado de desenvolvimento web utilizando o ecossistema React
		\item Estratégia de versionamento utilizando \href{https://nvie.com/posts/a-successful-git-branching-model/}{Git Flow} adaptado ao contexto do IF Salas
	\end{itemize}
\end{frame}

\begin{frame}{Metodologia}
	{Codificação e versionamento}
	\begin{figure}
		\includegraphics[width=9cm]{images/git-flow.png}
		\caption{Git Flow adaptado}
	\end{figure}
\end{frame}


\subsection{Implantação e Coleta de Dados}
\begin{frame}{Metodologia}
	{Implantação e Coleta de Dados}
	\begin{itemize}[<+->]
		\item Implantação automatizada através do \href{https://docs.github.com/en/actions}{GitHub Actions} para os domínios:
			\begin{itemize}
				\item \href{https://dev.ifsalas.xyz/}{https://dev.ifsalas.xyz/}
				\item \href{https://ifsalas.xyz/}{https://ifsalas.xyz/}
			\end{itemize}
		\item Coleta de feedbacks foi feita de maneira informal
	\end{itemize}
\end{frame}